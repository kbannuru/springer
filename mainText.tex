\section{Introduction} \label{sec:intro}

% Contributions of the paper
In this paper, we first...

% Organization of the paper
The remainder of this paper is organized as follows.
Section~\ref{sec:problem} formally defines our problem.
Section~\ref{sec:related} reviews the related works.
Section~\ref{sec:algorithms} presents the algorithms,   which are evaluated on Section~\ref{sec:experiments}. Finally, Section~\ref{sec:conclusions} draws the concluding remarks of our paper.

\section{The problem} \label{sec:problem}


\section{Related works} \label{sec:related}


\section{Algorithms} \label{sec:algorithms}

\subsection{Algorithm 1}

\subsection{Algorithm 2}

\subsection{Algorithm n}


\section{Computational experiments} \label{sec:experiments}

% Computational envoirnment description
The computational experiments have been done on a single core of an Intel
with x.x Ghz clock and x Gb of RAM,
unning underthe  operating  system  Linux  Ubuntu.
We used the ILOG  CPLEX  solver  version  12.8  with default parameters setting. 
The algorithms were implemented in C{}\verb!++! along with the ILOG Concert Technology and compiled with the GNU g{}\verb!++! 8.2.0. 
The running time of all algorithms has been set to 7200 seconds.


\subsection{Instance's description} \label{subsec:instances}


\subsection{Expriment 1}

\subsection{Experiment 2}

\subsection{Experiment m}


\section{Conclusions} \label{sec:conclusions}
